\documentclass{scrartcl}
\usepackage{cmap}
\usepackage[T1]{fontenc}
\usepackage{lmodern}
\usepackage[english]{babel}
\usepackage{amsmath}
\usepackage{booktabs, array}
\usepackage{tikz, environ}
\usepackage{pgffor, fp}
\usepackage{hyperref}

%%%%%%%%%
% SETUP %
%%%%%%%%%

\usetikzlibrary{decorations.pathreplacing, calligraphy, arrows.meta}

\DeclareMathOperator{\op}{op}
\DeclareMathOperator{\instr}{instr}

\makeatletter
\newsavebox{\measure@tikzpicture}
\NewEnviron{scalepicture}[1]{%
  \def\tikz@width{#1}%
  \def\tikzscale{1}\begin{lrbox}{\measure@tikzpicture}%
  \BODY
  \end{lrbox}%
  \pgfmathparse{#1/\wd\measure@tikzpicture}%
  \edef\tikzscale{\pgfmathresult}%
  \BODY
}
\makeatother

%%%%%%%%%%%%
% DOCUMENT %
%%%%%%%%%%%%

\title{Instruction set reference}
\author{%
  Jean \textsc{Caspar},
  Loïc \textsc{Chevalier},
  Vladimir \textsc{Ivanov},
  Adrien \textsc{Mathieu}
}
\date{}

\begin{document}
\maketitle{}
\section*{Hardware requirements}
This instruction set requires 16 registers, \(r_0,\ldots,r_{15}\). Each is a
general purpose 16 bit register, which can be accessed both for writing and for
reading, even in a single cycle, in which case the data read is the one stored
in the register \textit{before} the write operation. Every register must initially
be initialized to zeros.

\section*{Instructions}
An instruction \verb|instr| takes 32 bits. The first eight bits encode the
instruction code, the 24 remaining bits contain the arguments. See
the figure \nameref{fig:instrs} and the table \nameref{tab:instrs}.
\subsection*{Instruction code}
The first bit of an instruction code corresponds to the arity of the instruction.
A $0$ means an arity of one, a $1$ means an arity of two. Every instruction
additionally receives a \emph{destination register} $r_d$, in which the result of
the operation must be stored, unless told otherwise.\par
The next five bits encode to the operation. The first two bits indicate the
operation family:
\begin{itemize}
\item $00$: Arithmetic operation.
\item $01$: Jump operation.
\item $10$: RAM operation
\item $11$: Comparison operation.
\end{itemize}
The next three bits specify the actual operation performed.\par
The two final bits of the instruction code indicate whether arguments are
registers or immediates, $1$ meaning register and $0$ immediate. The first of
these two bits corresponds $a_1$, whereas the second to $a_2$.
\subsection*{Arguments}
Arguments are of two kind: immediates and registers. Registers require four bits,
whereas immediates are 16 bits signed integers. At most one immediate can be
given.\par
The first ``argument'' is always a register, $r_d$. The next argument $a_1$, or
the next two arguments $a_1$ and $a_2$, in this order, can be registers or
immediates.

\section*{Implementation details}
\subsection*{RAM operations}
\verb|load| and \verb|store| are encoded in the same way, but for the arity. This
allows merging them into a single operation, with the arity feeding the \verb|we|
flag of the RAM and register file modules.
\subsection*{Comparison operation}
The comparison operation's last three bits encore the actual comparison made.
In these three bits, the last two are used to distinguish between \(=\), \(<\)
and \(>\), whilst the first bit is used to negate the output. This allows only
implementing three actual comparisons, and \verb|xor| the result with the first
bit.
\subsection*{Jump operations}
Jump operations are already one-shot encoded, no further decoding is needed.
\begin{figure}[b]
  \label{fig:instrs}
  \begin{scalepicture}{\textwidth}
    \begin{tikzpicture}[
      thick,
      scale=\tikzscale,
      brace/.style={
        decorate,
        decoration={
          calligraphic brace,
          raise=2pt,
          amplitude=10pt,
        },
      }
      ]

      \draw[brace] (0,2) -- (8,2);
      \draw[brace] (8,2) -- (32,2);
      \node at (4,3.5) {instruction code};
      \node at (20,3.5) {arguments};

      %% 0, 1X
      
      \foreach \n in {0,...,31}{
        \FPeval{\next}{clip(\n+1)}
        \draw (\n,0) rectangle (\next,1);
        \node at (\n.5,1.5) {$\n$};
      }
      \foreach \n in {0,1,6,8,12,16}{
        \draw[dotted] (\n,2) -- (\n,-1);
      }
      \node at (3.5,-.5) {operation};
      \node at (10,-.5) {$r_d$};
      
      \node at (.5,.5) {$0$};
      \node at (6.5,.5) {$1$};
      \node at (7.5,.5) {X};
      \node at (14,-.5) {$r_1$};
      \foreach \n in {16,...,31}{
        \node at (\n.5,.5) {X};
      }

      %% 0, 0X
      \foreach \n in {0,...,31}{
        \FPeval{\next}{clip(\n+1)}
        \draw (\n,-2) rectangle (\next,-1);
      }
      \foreach \n in {0,1,6,8,12,28}{
        \draw[dotted] (\n,-3) -- (\n,-2);
      }
      \node at (3.5,-2.5) {operation};
      \node at (10,-2.5) {$r_d$};
      \node at (.5,-1.5) {$0$};
      
      \node at (6.5,-1.5) {$0$};
      \node at (7.5,-1.5) {X};
      \node at (20,-2.5) {$i$};
      \foreach \n in {28,...,31}{
        \node at (\n.5,-1.5) {X};
      }

      %% 1, 11
      \foreach \n in {0,...,31}{
        \FPeval{\next}{clip(\n+1)}
        \draw (\n,-4) rectangle (\next,-3);
      }
      \foreach \n in {0,1,6,8,12,16,20}{
        \draw[dotted] (\n,-5) -- (\n,-4);
      }
      \node at (3.5,-4.5) {operation};
      \node at (10,-4.5) {$r_d$};
      \node at (.5,-3.5) {$1$};
      
      \node at (6.5,-3.5) {$1$};
      \node at (7.5,-3.5) {$1$};
      \node at (14,-4.5) {$r_1$};
      \node at (18,-4.5) {$r_2$};
      \foreach \n in {20,...,31}{
        \node at (\n.5,-3.5) {X};
      }

      %% 1, 10
      \foreach \n in {0,...,31}{
        \FPeval{\next}{clip(\n+1)}
        \draw (\n,-6) rectangle (\next,-5);
      }
      \foreach \n in {0,1,6,8,12,16,32}{
        \draw[dotted] (\n,-7) -- (\n,-6);
      }
      \node at (3.5,-6.5) {operation};
      \node at (10,-6.5) {$r_d$};
      \node at (.5,-5.5) {$1$};
      
      \node at (6.5,-5.5) {$1$};
      \node at (7.5,-5.5) {$0$};
      \node at (14,-6.5) {$r_1$};
      \node at (24,-6.5) {$i$};

      %% 1, 01
      \foreach \n in {0,...,31}{
        \FPeval{\next}{clip(\n+1)}
        \draw (\n,-8) rectangle (\next,-7);
      }
      \foreach \n in {0,1,6,8,12,28,32}{
        \draw[dotted] (\n,-9) -- (\n,-8);
      }
      \node at (3.5,-8.5) {operation};
      \node at (10,-8.5) {$r_d$};
      \node at (.5,-7.5) {$1$};
      
      \node at (6.5,-7.5) {$0$};
      \node at (7.5,-7.5) {$1$};
      \node at (20,-8.5) {$i$};
      \node at (30,-8.5) {$r_2$};
    \end{tikzpicture}
  \end{scalepicture}
  \caption{Layout of instructions}
\end{figure}

\begin{table}
  \center
  \label{tab:instrs}
  \caption{Layout of instructions}
  \begin{tabular}{lcc}
    \toprule
    \textbf{Bits used} & \textbf{value} & \textbf{size}\\
    \midrule
    \multicolumn{3}{c}{\(r_d = \op(r_1)\)}\\
    \(\instr[0:0]\) & \(0\) & 1 bit\\
    \(\instr[5:1]\) & \(\op\) & 5 bits\\
    \(\instr[6:6]\) & \(1\) & 1 bit\\
    \(\instr[11:8]\) & \(r_d\) & 4 bits\\
    \(\instr[15:12]\) & \(r_1\) & 4 bits\\
    \midrule
    \multicolumn{3}{c}{\(r_d = \op(i)\)}\\
    \(\instr[0:0]\) & \(0\) & 1 bit\\
    \(\instr[5:2]\) & \(\op\) & 5 bits\\
    \(\instr[6:6]\) & \(0\) & 1 bit\\
    \(\instr[11:8]\) & \(r_d\) & 4 bits\\
    \(\instr[27:12]\) & \(i\) & 16 bits\\
    \midrule
    \multicolumn{3}{c}{\(r_d = \op(r_1, r_2)\)}\\
    \(\instr[0:0]\) & \(1\) & 1 bit\\
    \(\instr[5:1]\) & \(\op\) & 5 bits\\
    \(\instr[7:6]\) & \(11\) & 2 bits\\
    \(\instr[11:8]\) & \(r_d\) & 4 bits\\
    \(\instr[15:12]\) & \(r_1\) & 4 bits\\
    \(\instr[19:16]\) & \(r_2\) & 4 bits\\
    \midrule
    \multicolumn{3}{c}{\(r_d = \op(i, r_2)\)}\\
    \(\instr[0:0]\) & \(1\) & 1 bits\\
    \(\instr[5:1]\) & \(\op\) & 5 bits\\
    \(\instr[7:6]\) & \(01\) & 2 bits\\
    \(\instr[11:8]\) & \(r_d\) & 4 bits\\
    \(\instr[27:12]\) & \(i\) & 16 bits\\
    \(\instr[31:28]\) & \(r_2\) & 4 bits\\
    \midrule
    \multicolumn{3}{c}{\(r_d = \op(r_1, i)\)}\\
    \(\instr[0:0]\) & \(1\) & 1 bit\\
    \(\instr[5:1]\) & \(\op\) & 5 bits\\
    \(\instr[7:6]\) & \(10\) & 2 bits\\
    \(\instr[11:8]\) & \(r_d\) & 4 bits\\
    \(\instr[15:12]\) & \(r_1\) & 4 bits\\
    \(\instr[31:16]\) & \(i\) & 16 bits\\
    \bottomrule
  \end{tabular}
\end{table}

\begin{table}
  \caption{Operation list}

  \begin{tabular}{lcrlr}
    \toprule
    \textbf{Name} & \textbf{Arity} & \textbf{Encoding} & \textbf{Effect}\\
    \midrule
    \midrule
    \multicolumn{3}{c}{Arithmetic operation}\\
    \midrule
    add & 2 & \(00\;000\) & \(r_d=a_1+a_2\)\\
    sub & 2 & \(00\;001\) & \(r_d=a_1-a_2\)\\
    mul & 2 & \(00\;010\) & \(r_d=a_1\cdot a_2\)\\
    div & 2 & \(00\;011\) & \(r_d=a_1/a_2\)\\
    mod & 2 & \(00\;100\) & \(r_d=a_1\% a_2\)\\
    and & 2 & \(00\;101\) & \(r_d=a_1 \& a_2\)\\
    or & 2 & \(00\;110\) & \(r_d=a_1 | a_2\)\\
    xor & 2 & \(00\;111\) & \(r_d=a_1\bigoplus a_2\)\\
    \midrule
    \multicolumn{3}{c}{Jump operation}\\
    \midrule
    jmp & 1 & \(01\;000\) & \(ip=a_1\)\\
    jo & 1 & \(01\;001\) & \(ip=ip+a_1\)\\
    jz & 2 & \(01\;010\) & if \(a_1=0\), \(ip=a_2\)\\
    jzo & 2 & \(01\;011\) & if \(a_1=0\), \(ip=ip+a_2\)\\
    jnz & 2 & \(01\;101\) & if \(a_1\neq0\), \(ip=a_2\)\\
    jnzo & 
    \midrule
    \multicolumn{3}{c}{Comparison operation}\\
    \midrule
    cmp\textit{cc} & 2 & \(11\;???\) & if \(a_1\;cc\;a_2\), \(r_d=1\),\\
                  &  &  & otherwise \(r_d=0\)\\
    \midrule
    \multicolumn{3}{c}{RAM operation}\\
    \midrule
    load & 1 & \(10\;000\) & \(r_d=\mathrm{RAM}[a_1]\)\\
    store & 2 & \(10\;000\) & \(\mathrm{RAM}[a_1]=a_2\)\\
    \bottomrule
  \end{tabular}
  \quad
  \begin{tabular}{lc}
    \multicolumn{2}{c}{\(cc\) encoding}\\
    \toprule
    \textbf{Name} & \textbf{Encoding}\\
    \midrule
    eq & \(000\)\\
    neq & \(100\)\\
    lt & \(001\)\\
    leq & \(110\)\\
    gt & \(010\)\\
    geq & \(101\)\\
    \bottomrule
  \end{tabular}

\end{table}
\end{document}
