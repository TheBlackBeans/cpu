\documentclass{scrartcl}
\usepackage{cmap}
\usepackage[T1]{fontenc}
\usepackage{lmodern}
\usepackage[english]{babel}
\usepackage{amsmath}
\usepackage{booktabs, array}
\usepackage{tikz, environ}
\usepackage{pgffor, fp}
\usepackage{hyperref}

%%%%%%%%%
% SETUP %
%%%%%%%%%

\usetikzlibrary{decorations.pathreplacing, calligraphy, arrows.meta}

\DeclareMathOperator{\op}{op}
\DeclareMathOperator{\instr}{instr}

\makeatletter
\newsavebox{\measure@tikzpicture}
\NewEnviron{scalepicture}[1]{%
  \def\tikz@width{#1}%
  \def\tikzscale{1}\begin{lrbox}{\measure@tikzpicture}%
  \BODY
  \end{lrbox}%
  \pgfmathparse{#1/\wd\measure@tikzpicture}%
  \edef\tikzscale{\pgfmathresult}%
  \BODY
}
\makeatother

%%%%%%%%%%%%
% DOCUMENT %
%%%%%%%%%%%%

\title{Instruction set reference}
\author{%
  Jean \textsc{Caspar},
  Loïc \textsc{Chevalier},
  Vladimir \textsc{Ivanov},
  Adrien \textsc{Mathieu}
}
\date{}

\begin{document}
\maketitle{}
\section*{Hardware requirements}
This instruction set requires 16 accessible registers, \(r_0,\ldots,r_{15}\), as
well as an instruction pointer \(ip\) and a RAM.\par
The first register \(r_0\) can be used the destination of an operation, in which
case the result of the aforementioned operation is ignored. It can also be used
as a source register, in which can it acts as the constant \(0\).\par
All the other registers are general purpose 16 bit registers, which can be
accessed both for writing and for reading, even in a single cycle.
The data read is the one stored in the register \textit{before} the write
operation. Initially, every register must be set to zero, as should the
RAM and the instruction pointer.\par
The size of the RAM is left unspecified, and the instruction pointer must have
the same size of the general registers.

\section*{Instructions}
An instruction is 32 bits wide. The first eight bits encode the
instruction code, the 24 remaining bits contain the arguments. See
the figure \nameref{fig:instrs} and the table \nameref{tab:instrs}.
\subsection*{Instruction code}
The first bit of an instruction code corresponds to the arity of the instruction.
A $0$ means an arity of one, a $1$ means an arity of two. Every instruction
additionally receives a \emph{destination register} $r_d$, in which the result of
the operation must be stored, unless told otherwise.\par
The next five bits encode to the operation. The first two bits indicate the
operation family:
\begin{itemize}
\item $00$: Arithmetic and logic operation.
\item $01$: Jump operation.
\item $10$: RAM operation
\item $11$: Comparison operation.
\end{itemize}
The next three bits specify the actual operation performed.\par
The two final bits of the instruction code indicate whether arguments are
registers or immediates, $1$ meaning register and $0$ immediate. The first of
these two bits corresponds to $a_1$, whereas the second to $a_2$.
\subsection*{Arguments}
Arguments are of two kind: immediates and registers. Registers require four bits,
whereas immediates are 16 bits signed integers. At most one immediate can be
given.\par
The first ``argument'' is always a register, $r_d$. The next argument $a_1$, or
the next two arguments $a_1$ and $a_2$, in this order, can be registers or
immediates.

\section*{Implementation details}
\subsection*{Miscellaneous operations}
\verb|load| and \verb|store|, as well as \verb|recv| and \verb|send|, are encoded
in the same way, except for the arity. This allows merging them into a single
operation, with the arity feeding the \verb|we| flag of the RAM/port and register
file modules.
\subsection*{Comparison operation}
The comparison operation's last three bits encore the actual comparison made.
The first bit indicates the negation of the result, the second indicates whether
the arguments are considered as signed, the last number indicates whether we
apply less than.
\begin{figure}[b]
  \label{fig:instrs}
  \begin{scalepicture}{\textwidth}
    \begin{tikzpicture}[
      thick,
      scale=\tikzscale,
      brace/.style={
        decorate,
        decoration={
          calligraphic brace,
          raise=2pt,
          amplitude=10pt,
        },
      }
      ]

      \draw[brace] (0,2) -- (24,2);
      \draw[brace] (24,2) -- (32,2);
      \node at (28,3.5) {instruction code};
      \node at (12,3.5) {arguments};

      %% 0, 1X
      
      \foreach \n in {0,...,31}{
        \FPeval{\next}{clip(\n+1)}
        \FPeval{\lab}{clip(31-\n)}
        \draw (\n,0) rectangle (\next,1);
        \node at (\n.5,1.5) {$\lab$};
      }
      \foreach \n in {16, 20, 24, 26, 31, 32}{
        \draw[dotted] (\n,0) -- (\n,-1);
      }
      \node at (28.5,-.5) {operation};
      \node at (22,-.5) {$r_d$};
      
      \node at (31.5,.5) {$0$};
      \node at (25.5,.5) {$1$};
      \node at (24.5,.5) {X};
      \node at (18,-.5) {$r_1$};
      \foreach \n in {0,...,15}{
        \node at (\n.5,.5) {X};
      }

      %% 0, 0X
      \foreach \n in {0,...,31}{
        \FPeval{\next}{clip(\n+1)}
        \draw (\n,-2) rectangle (\next,-1);
      }
      \foreach \n in {32,31,26,24,20,4}{
      % \foreach \n in {0,1,6,8,12,28}{
        \draw[dotted] (\n,-3) -- (\n,-2);
      }
      \node at (28.5,-2.5) {operation};
      \node at (22,-2.5) {$r_d$};
      \node at (31.5,-1.5) {$0$};
      
      \node at (25.5,-1.5) {$0$};
      \node at (24.5,-1.5) {X};
      \node at (12,-2.5) {$i$};
      \foreach \n in {0,...,3}{
        \node at (\n.5,-1.5) {X};
      }

      %% 1, 11
      \foreach \n in {0,...,31}{
        \FPeval{\next}{clip(\n+1)}
        \draw (\n,-4) rectangle (\next,-3);
      }
      \foreach \n in {32,31,26,24,20,16,4}{
        \draw[dotted] (\n,-5) -- (\n,-4);
      }
      \node at (28.5,-4.5) {operation};
      \node at (22,-4.5) {$r_d$};
      \node at (31.5,-3.5) {$1$};
      
      \node at (25.5,-3.5) {$1$};
      \node at (24.5,-3.5) {$1$};
      \node at (18,-4.5) {$r_1$};
      \node at (2,-4.5) {$r_2$};
      \foreach \n in {4,...,15}{
        \node at (\n.5,-3.5) {X};
      }

      %% 1, 10
      \foreach \n in {0,...,31}{
        \FPeval{\next}{clip(\n+1)}
        \draw (\n,-6) rectangle (\next,-5);
      }
      \foreach \n in {32,31,26,24,20,16,0}{
        \draw[dotted] (\n,-7) -- (\n,-6);
      }
      \node at (28.5,-6.5) {operation};
      \node at (22,-6.5) {$r_d$};
      \node at (31.5,-5.5) {$1$};
      
      \node at (25.5,-5.5) {$1$};
      \node at (24.5,-5.5) {$0$};
      \node at (18,-6.5) {$r_1$};
      \node at (8,-6.5) {$i$};

      %% 1, 01
      \foreach \n in {0,...,31}{
        \FPeval{\next}{clip(\n+1)}
        \draw (\n,-8) rectangle (\next,-7);
      }
      \foreach \n in {32,31,26,24,20,4,0}{
        \draw[dotted] (\n,-9) -- (\n,-8);
      }
      \node at (28.5,-8.5) {operation};
      \node at (22,-8.5) {$r_d$};
      \node at (31.5,-7.5) {$1$};
      
      \node at (25.5,-7.5) {$0$};
      \node at (24.5,-7.5) {$1$};
      \node at (12,-8.5) {$i$};
      \node at (2,-8.5) {$r_2$};
    \end{tikzpicture}
  \end{scalepicture}
  \caption{Layout of instructions}
\end{figure}

\begin{table}
  \center
  \label{tab:instrs}
  \caption{Layout of instructions}
  \begin{tabular}{lcc}
    \toprule
    \textbf{Bits used} & \textbf{value} & \textbf{size}\\
    \midrule
    \multicolumn{3}{c}{\(r_d = \op(r_1)\)}\\
    \(\instr[0:0]\) & \(0\) & 1 bit\\
    \(\instr[5:1]\) & \(\op\) & 5 bits\\
    \(\instr[6:6]\) & \(1\) & 1 bit\\
    \(\instr[11:8]\) & \(r_d\) & 4 bits\\
    \(\instr[15:12]\) & \(r_1\) & 4 bits\\
    \midrule
    \multicolumn{3}{c}{\(r_d = \op(i)\)}\\
    \(\instr[0:0]\) & \(0\) & 1 bit\\
    \(\instr[5:1]\) & \(\op\) & 5 bits\\
    \(\instr[6:6]\) & \(0\) & 1 bit\\
    \(\instr[11:8]\) & \(r_d\) & 4 bits\\
    \(\instr[27:12]\) & \(i\) & 16 bits\\
    \midrule
    \multicolumn{3}{c}{\(r_d = \op(r_1, r_2)\)}\\
    \(\instr[0:0]\) & \(1\) & 1 bit\\
    \(\instr[5:1]\) & \(\op\) & 5 bits\\
    \(\instr[7:6]\) & \(11\) & 2 bits\\
    \(\instr[11:8]\) & \(r_d\) & 4 bits\\
    \(\instr[15:12]\) & \(r_1\) & 4 bits\\
    \(\instr[31:28]\) & \(r_2\) & 4 bits\\
    \midrule
    \multicolumn{3}{c}{\(r_d = \op(i, r_2)\)}\\
    \(\instr[0:0]\) & \(1\) & 1 bits\\
    \(\instr[5:1]\) & \(\op\) & 5 bits\\
    \(\instr[7:6]\) & \(01\) & 2 bits\\
    \(\instr[11:8]\) & \(r_d\) & 4 bits\\
    \(\instr[27:12]\) & \(i\) & 16 bits\\
    \(\instr[31:28]\) & \(r_2\) & 4 bits\\
    \midrule
    \multicolumn{3}{c}{\(r_d = \op(r_1, i)\)}\\
    \(\instr[0:0]\) & \(1\) & 1 bit\\
    \(\instr[5:1]\) & \(\op\) & 5 bits\\
    \(\instr[7:6]\) & \(10\) & 2 bits\\
    \(\instr[11:8]\) & \(r_d\) & 4 bits\\
    \(\instr[15:12]\) & \(r_1\) & 4 bits\\
    \(\instr[31:16]\) & \(i\) & 16 bits\\
    \bottomrule
  \end{tabular}
\end{table}

\begin{table}
  \caption{Operation list}

  \begin{tabular}{lcrl}
    \toprule
    \textbf{Name} & \textbf{Arity} & \textbf{Encoding} & \textbf{Effect}\\
    \midrule
    \midrule
    \multicolumn{3}{c}{Arithmetic and logic operation}\\
    \midrule
    add & 2 & \(00\;000\) & \(r_d=a_1+a_2\)\\
    sub & 2 & \(00\;001\) & \(r_d=a_1-a_2\)\\
    mul & 2 & \(00\;010\) & \(r_d=a_1\cdot a_2\)\\
    div & 2 & \(00\;011\) & \(r_d=a_1/a_2\)\\
    mod & 2 & \(00\;100\) & \(r_d=a_1\% a_2\)\\
    and & 2 & \(00\;101\) & \(r_d=a_1 \& a_2\)\\
    or & 2 & \(00\;110\) & \(r_d=a_1 | a_2\)\\
    xor & 2 & \(00\;111\) & \(r_d=a_1\bigoplus a_2\)\\
    \midrule
    \multicolumn{3}{c}{Jump operation}\\
    \midrule
    jmp & 1 & \(01\;000\) & \(ip=a_1\)\\
    jo & 1 & \(01\;001\) & \(ip=ip+a_1\)\\
    jz & 2 & \(01\;000\) & if \(a_2=0\), \(ip=a_1\)\\
    jzo & 2 & \(01\;001\) & if \(a_2=0\), \(ip=ip+a_1\)\\
    jnz & 2 & \(01\;010\) & if \(a_2\neq0\), \(ip=a_1\)\\
    jnzo & 2 & \(01\;011\) & if \(a_2\neq0\), \(ip=ip+a_1\)\\
    \midrule
    \multicolumn{3}{c}{Comparison operation}\\
    \midrule
    cmp\textit{cc} & 2 & \(11\;???\) & if \(a_1\;cc\;a_2\), \(r_d=1\),\\
                  &  &  & otherwise \(r_d=0\)\\
    \midrule
    \multicolumn{3}{c}{Miscellaneous}\\
    \midrule
    load & 1 & \(10\;000\) & \(r_d=\mathrm{RAM}[a_1]\)\\
    store & 2 & \(10\;000\) & \(\mathrm{RAM}[a_1]=a_2\)\\
    recv & 1 & \(10\;001\) & \(r_d=\mathrm{port}~a_1\)\\
    send & 2 & \(10\;001\) & \(\mathrm{port}~a_1=a_2\)\\
    \bottomrule
  \end{tabular}
  \quad
  \begin{tabular}{lc}
    \multicolumn{2}{c}{\(cc\) encoding}\\
    \toprule
    \textbf{Name} & \textbf{Encoding}\\
    \midrule
    eq & \(000\)\\
    ne & \(100\)\\
    bl & \(001\)\\
    ae & \(101\)\\
    lt & \(011\)\\
    ge & \(111\)\\
    \bottomrule
  \end{tabular}

\end{table}
\end{document}
