\documentclass{scrartcl}
\usepackage{cmap}
\usepackage[T1]{fontenc}
\usepackage{lmodern}
\usepackage[english]{babel}
\usepackage{amsmath}
\usepackage{booktabs, array}

\DeclareMathOperator{\op}{op}
\DeclareMathOperator{\instr}{instr}

\begin{document}
\section*{High level description}
\subsection*{Registers}
There are 16 registers, \(r_0,\ldots,r_{15}\). Each is a general purpose 16 bit
register.
\subsection*{Immediates}
All immediates are 16 bit signed integers.

\section*{Instructions}
\subsection*{Instruction description}
An instruction \verb|instr| takes 32 bits. An instruction has (up to) four
components. The first eight bits are always the operation performed. If the
operation takes at least one argument, it is encoded on the next four bits.\par
An instruction cannot have two immediate arguments.
\begin{table}[h]
  \caption{Layout of instructions}
  \begin{minipage}{0.5\textwidth}
    \begin{tabular}{lcc}
      \toprule
      \textbf{Bits used} & \textbf{value} & \textbf{size}\\
      \midrule
      \multicolumn{3}{c}{\(\op()\)}\\
      \(\instr[1:0]\) & \(arity\) & 2 bits\\
      \(\instr[5:2]\) & \(\op\) & 4 bits\\
      \midrule
      \multicolumn{3}{c}{\(r_d = \op()\)}\\
      \(\instr[1:0]\) & \(arity\) & 2 bits\\
      \(\instr[5:2]\) & \(\op\) & 4 bits\\
      \(\instr[11:8]\) & \(r_d\) & 4 bits\\
      \midrule
      \multicolumn{3}{c}{\(r_d = \op(r_1)\)}\\
      \(\instr[1:0]\) & \(arity\) & 2 bits\\
      \(\instr[5:2]\) & \(\op\) & 4 bits\\
      \(\instr[6:6]\) & \(1\) & 1 bit\\
      \(\instr[11:8]\) & \(r_d\) & 4 bits\\
      \(\instr[15:12]\) & \(r_1\) & 4 bits\\
      \midrule
      \multicolumn{3}{c}{\(r_d = \op(i)\)}\\
      \(\instr[1:0]\) & \(arity\) & 2 bits\\
      \(\instr[5:2]\) & \(\op\) & 4 bits\\
      \(\instr[6:6]\) & \(0\) & 1 bit\\
      \(\instr[11:8]\) & \(r_d\) & 4 bits\\
      \(\instr[27:12]\) & \(i\) & 16 bits\\
      \bottomrule
    \end{tabular}
  \end{minipage}
  \begin{minipage}{0.5\textwidth}
    \begin{tabular}{lcc}
      \toprule
      \textbf{Bits used} & \textbf{value} & \textbf{size}\\
      \midrule
      \multicolumn{3}{c}{\(r_d = \op(r_1, r_2)\)}\\
      \(\instr[1:0]\) & \(arity\) & 2 bits\\
      \(\instr[5:2]\) & \(\op\) & 4 bits\\
      \(\instr[7:6]\) & \(11\) & 2 bits\\
      \(\instr[11:8]\) & \(r_d\) & 4 bits\\
      \(\instr[15:12]\) & \(r_1\) & 4 bits\\
      \(\instr[19:16]\) & \(r_2\) & 4 bits\\
      \midrule
      \multicolumn{3}{c}{\(r_d = \op(i, r_2)\)}\\
      \(\instr[1:0]\) & \(arity\) & 2 bits\\
      \(\instr[5:2]\) & \(\op\) & 4 bits\\
      \(\instr[7:6]\) & \(01\) & 2 bits\\
      \(\instr[11:8]\) & \(r_d\) & 4 bits\\
      \(\instr[27:12]\) & \(i\) & 16 bits\\
      \(\instr[31:28]\) & \(r_2\) & 4 bits\\
      \midrule
      \multicolumn{3}{c}{\(r_d = \op(r_1, i)\)}\\
      \(\instr[1:0]\) & \(arity\) & 2 bits\\
      \(\instr[5:2]\) & \(\op\) & 4 bits\\
      \(\instr[7:6]\) & \(10\) & 2 bits\\
      \(\instr[11:8]\) & \(r_d\) & 4 bits\\
      \(\instr[15:12]\) & \(r_1\) & 4 bits\\
      \(\instr[31:16]\) & \(i\) & 16 bits\\
      \bottomrule
    \end{tabular}
  \end{minipage}
\end{table}
\subsection*{Instruction list}
\begin{table}
  \center
  \caption{Operation list}
  \begin{tabular}{lcrlr}
    \toprule
    \textbf{Name} & \textbf{Arity} & \textbf{Encoding} & \textbf{Effect} & \textbf{Notes}\\
    \midrule
    add & 2 & \(0000\) & \(r_d=a_1+a_2\)\\
    sub & 2 & \(0001\) & \(r_d=a_1-a_2\)\\
    mul & 2 & \(0010\) & \(r_d=a_1\cdot a_2\)\\
    div & 2 & \(0011\) & \(r_d=a_1/a_2\)\\
    mod & 2 & \(0100\) & \(r_d=a_1\% a_2\)\\
    and & 2 & \(0101\) & \(r_d=a_1 \& a_2\)\\
    or & 2 & \(0110\) & \(r_d=a_1 | a_2\)\\
    xor & 2 & \(0111\) & \(r_d=a_1\bigoplus a_2\)\\
    jmp & 1 & \(0001\) & \(ip=a_1\) & \(r_d\) is ignored\\
    jo & 1 & \(0010\) & \(ip=ip+a_1\) & \(r_d\) is ignored\\
    jz & 2 & \(1000\) & if \(a_1=0\), \(ip=a_2\)\\
    cmp\textit{cc} & 2 & \(1001\) & if \(a_1\;cc\;a_2\), \(r_d=1\), otherwise \(r_d=0\)\\
    load & 1 & \(0011\) & \(r_d=\mathrm{RAM}[a_1]\)\\
    store & 2 & \(1010\) & \(\mathrm{RAM}[a_1]=a_2\) & \(r_d\) is ignored\\
    \bottomrule
  \end{tabular}
\end{table}
% \subsubsection*{Instruction without arguments}
% \begin{description}
% \item[\verb+instr[7:0]+] The first component encodes the operation performed, on
%   $8$ bits.
% \end{description}
% \subsubsection*{Instruction with a destination}
% \begin{description}
% \item[\verb|instr[7:0]|] The first component encodes the operation performed, on
%   $8$ bits.
% \item[\verb|instr[11:8]|] The second component encodes the destination register,
%   on $4$ bits.
% \end{description}
% \subsubsection*{Instruction with a destination, and a register argument}
% \begin{description}
% \item[\verb|instr[7:0]|] The first component encodes the operation performed, on
%   $8$ bits.
% \item[\verb|instr[11:8]|] The second component encodes the destination register,
%   on $4$ bits.
% \item[\verb|instr[15:12]|] The third component encodes the first argument register,
%   on $4$ bits.
% \end{description}
% \subsubsection*{Instruction with a destination, and two register arguments}
% \begin{description}
% \item[\verb|instr[7:0]|] The first component encodes the operation performed, on
%   $8$ bits.
% \item[\verb|instr[11:8]|] The second component encodes the destination register,
%   on $4$ bits.
% \item[\verb|instr[15:12]|] The third component encodes the first argument register,
%   on $4$ bits.
% \item[\verb|instr[19:16]|] The forth component encodes the second argument register,
%   on $4$ bits.
% \end{description}
% \subsubsection*{Instruction with a destination, an immediate argument then a register argument}
% \begin{description}
% \item[\verb|instr[7:0]|] The first component encodes the operation performed, on
%   $8$ bits.
% \item[\verb|instr[11:8]|] The second component encodes the destination register,
%   on $4$ bits.
% \item[\verb|instr[27:12]|] The third component encodes the immediate argument,
%   on $16$ bits.
% \item[\verb|instr[31:28]|] The forth component encodes the second argument register,
%   on $4$ bits.
% \end{description}
% \subsubsection*{Instruction with a destination, a register argument then an immediate argument}
% \begin{description}
% \item[\verb|instr[7:0]|] The first component encodes the operation performed, on
%   $8$ bits.
% \item[\verb|instr[11:8]|] The second component encodes the destination register,
%   on $4$ bits.
% \item[\verb|instr[15:12]|] The third component encodes the second argument register,
%   on $4$ bits.
% \item[\verb|instr[31:16]|] The forth component encodes the immediate argument,
%   on $16$ bits.
% \end{description}

\end{document}
